\subsection{External Interface Requirements}
\subsubsection{User Interfaces}

\begin{itemize}
    \item \textbf{Web Application Interface}

%Log in and sing up on the Web App
\begin{figure}[H]
  \centering
   \includegraphics[width=0.6\textwidth]{Images/LoginWebApp}
  \caption{{Log in and sing up on the Web Application}}
  \label{fig:LoginWebApp}
\end{figure}

%Log in screen of the Web App
\begin{figure}[H]
  \centering
   \includegraphics[width=0.6\textwidth]{Images/LogInBox.png}
  \caption{Log in screen of the Web Application}
  \label{fig:LoginScreen}
\end{figure}

%Principal page of the Web App
\begin{figure}[H]
  \centering
   \includegraphics[width=0.6\textwidth]{Images/PrincipalPageWebApp2}
  \caption{Principal page of the Web Application}
  \label{fig:PrincipalPageWebApp}
\end{figure}

%Search for individual data in the Web App
\begin{figure}[H]
  \centering
   \includegraphics[width=0.6\textwidth]{Images/SearchIndividualWebApp}
  \caption{Search for individual data in the Web Application}
  \label{fig:SearchIndividualWebApp}
\end{figure}

%Result of the individual search in the Web App
\begin{figure}[H]
  \centering
   \includegraphics[width=0.6\textwidth]{Images/ResultIndividualSearch1}
  \caption{Results of the individual search in the Web Application (First part)}
  \label{fig:ResultIndividualSearch1}
\end{figure}

%Result of the individual search in the Web App
\begin{figure}[H]
  \centering
   \includegraphics[width=0.6\textwidth]{Images/ResultIndividualSearch2}
  \caption{Results of the individual search in the Web Application (Second part)}
  \label{fig:ResultIndividualSearch2}
\end{figure}

%Result of the individual search in the Web App
\begin{figure}[H]
  \centering
   \includegraphics[width=0.6\textwidth]{Images/ResultIndividualSearch3}
  \caption{Results of the individual search in the Web Application (Third part)}
  \label{fig:ResultIndividualSearch3}
\end{figure}

%Group search in the Web App
\begin{figure}[H]
  \centering
   \includegraphics[width=0.6\textwidth]{Images/GrupalSearchScreen}
  \caption{Search for group data in the Web Application}
  \label{fig:GroupSearch}
\end{figure}

%Result group search in the Web App (First part)
\begin{figure}[H]
  \centering
   \includegraphics[width=0.6\textwidth]{Images/ResultGroupSearch}
  \caption{Result of the group search in the Web Application (First part)}
  \label{fig:ResultGroupSearch}
\end{figure}

%Result group search in the Web App (Second part)
\begin{figure}[H]
  \centering
   \includegraphics[width=0.6\textwidth]{Images/ResultGroupSearch2}
  \caption{Result of the group search in the Web Application (Second part)}
  \label{fig:ResultGroupSearch2}
\end{figure}

%Result group search in the Web App (Third part)
\begin{figure}[H]
  \centering
   \includegraphics[width=0.6\textwidth]{Images/ResultGroupSearch3}
  \caption{Result of the group search in the Web Application (Third part)}
  \label{fig:ResultGroupSearch3}
\end{figure}

%Contact zone of the Web App
\begin{figure}[H]
  \centering
   \includegraphics[width=0.6\textwidth]{Images/ContactWebApp}
  \caption{Contact zone of the web application}
  \label{fig:ContactWebApp}
\end{figure}

\item \textbf{Application Interface}

\begin{figure} [H]
\centering
  \begin{minipage}{0.4\textwidth}
    \centering
    \includegraphics[width=0.7\textwidth]{Images/LogInApp}
    \caption{Log in and sing up window of the application}
    \label{fig:LogInScreenApp}
  \end{minipage}%
  \hspace{5mm}
  \begin{minipage}{0.4\textwidth}
    \centering
    \includegraphics[width=0.7\textwidth]{Images/PrincipalScreenApp}
    \caption{Main window of the application}
    \label{fig:PrincipalScreenApp}
  \end{minipage}
\end{figure}

\begin{figure}[H]
  \centering
   \includegraphics[width=0.3\textwidth]{Images/PrincipalScreenApp2}
  \caption{Main application window with slider tab}
  \label{fig:PrincipalScreenAppWithSlide}
\end{figure}

\item \textbf{Smartwatch Interface}

\begin{figure}[H]
  \centering
   \includegraphics[width=0.55\textwidth]{Images/SmartwatchScreen.png}
  \caption{Smartwatch window}
  \label{fig:SmartwatchWindow}
\end{figure}

\end{itemize}

\subsubsection{Software Interfaces}\label{sec:interfaces}
In this section, the API offered by \companyName{} will be detailed. This API is intended to be use for the clients who seeks an integration with their own systems.
\begin{table}[H]
  \centering
    \begin{adjustbox}{max width=\textwidth}
  \begin{tabular}{cccp{4cm}p{2cm}p{9cm}}
    \toprule
    \textbf{ID} & \textbf{Method} & \textbf{URL} & \textbf{Parameters} & \textbf{Return} & \textbf{Description} \\
    \midrule
    
    \silabel{si:api-login} & POST &
    /login
    &
    \begin{tabular}{@{}p{4cm}@{}}
    Username, password
    \end{tabular} &
    \begin{tabular}{@{}p{2cm}@{}}
    Client token
    \end{tabular} &
    \begin{tabular}{@{}p{9cm}@{}}
    Returns a token that will be use to authenticate the user in the next API calls.
    \end{tabular} \\
    \hline
    
    \silabel{si:api-query} & POST &
    /query
    &
    \begin{tabular}{@{}p{4cm}@{}}
    Client token, query
    \end{tabular} &
    \begin{tabular}{@{}p{2cm}@{}}
    Data batch
    \end{tabular} &
    \begin{tabular}{@{}p{9cm}@{}}
    Return a data batch containing all the entries that corresponds to the query.
    \end{tabular} \\
    \hline
    
    \silabel{si:api-search} & POST &
    /search-user
    &
    \begin{tabular}{@{}p{4cm}@{}}
    Client token, User ID
    \end{tabular} &
    \begin{tabular}{@{}p{2cm}@{}}
    Data of the user
    \end{tabular} &
    \begin{tabular}{@{}p{9cm}@{}}
    Returns the available data of the searched user if the client has permission
    \end{tabular} \\
    \bottomrule
  \end{tabular}
  \end{adjustbox}
  \caption{Software interfaces of \companyName{} API for the client}
  \label{table:api-client}
\end{table}

\subsubsection{Hardware Interfaces}

\companyName{} is presented as a software that is based on an Android application, because of it does not require special hardware other than the one named above, this hardware is more than anything a mobile device such as a Smartphone or a Smartwach. As discussed below it is necessary that the mobile device has an Internet connection as well as Bluethoot (if Smartwatch is used) and GPS (for tracking the location).

The system will run on any off-the-shelf hardware that supports Linux x64. Therefore, no special requirements and interfaces are needed. The relevant constraints for the users' devices are stated in section \ref{sec:constraints}.

\subsection{Use cases}
This section is meant to clarify the normal usage of the platform by the different actors of the system.

\begin{table}[H]
  \centering
    \begin{adjustbox}{max width=\textwidth}
  \begin{tabular}{rp{14cm}}
    \toprule
    \textbf{ID} & \uselabel{use:UseCaseSingUpClient}\\
    \hline
    \textbf{Name} & Sing up of Clients.\\
    \hline
    \textbf{Actor} & Client.\\
    \hline
    \textbf{Entry conditions} & The client have to be on the web application.\\
    \hline
    \textbf{Events flow} &
    \begin{tabular}{@{}p{14cm}@{}}
        \begin{enumerate}[leftmargin=*]
            \item The client have to click on the button of sing up of the web application (figure \ref{fig:LoginWebApp}).
            \item Fill in the necessary information requested in the form that will appear as well as a form of payment.
            \item After the confirmation the system will save the data and the client will be registered.
        \end{enumerate} 
    \end{tabular} \\ 
    \hline
    \textbf{Exit conditions} & The client will be registered and able to work with \companyName{}.\\
   
    \hline
    \textbf{Exceptions} & 
     \begin{tabular}{@{}p{14cm}@{}}
        \begin{enumerate}[leftmargin=*]
            \item The form of payment is not accepted or is incorrect.
            \item The client is already registered.
            \item The client has not filled in one of the necessary information fields or a field is not filled in correctly.
        \end{enumerate} 
    \end{tabular} \\
    
    \bottomrule
  \end{tabular}
  \end{adjustbox}
  \caption{Sing up of a Client use case}
  \label{table: SingUpClient}
\end{table}

\begin{table}[H]
  \centering
    \begin{adjustbox}{max width=\textwidth}
  \begin{tabular}{rp{14cm}}
    \toprule
    \textbf{ID} & \uselabel{use:UseCaseLogInClient}\\
    \hline
    \textbf{Name} & Log in of Clients.\\
    \hline
    \textbf{Actor} & Client.\\
    \hline
    \textbf{Entry conditions} & 
    \begin{tabular}{@{}p{14cm}@{}}
         \begin{enumerate}
             \item The client have to be registered on \companyName{}.
             \item The client have to be on the web application.
         \end{enumerate}  \\
    \end{tabular}\\
    \hline
    \textbf{Events flow} &
    \begin{tabular}{@{}p{14cm}@{}}
        \begin{enumerate}[leftmargin=*]
            \item Press the log in button in the web application (figure \ref{fig:LoginWebApp}).
            \item Complete the username and password sections of the log in window (figure \ref{fig:LoginScreen}).
            \item After clicking on the log in button, if it is correct the username and password will access to user account and the system will redirect to the main window (figure \ref{fig:PrincipalPageWebApp}).
        \end{enumerate} 
    \end{tabular} \\ 
    \hline
    \textbf{Exit conditions} & The client will access his account.\\
   
    \hline
    \textbf{Exceptions} & Username and password do not match or do not exist.\\
    
    \bottomrule
  \end{tabular}
  \end{adjustbox}
  \caption{Log in of a Client use case}
  \label{table:UseCaseLogInClient}
\end{table}

\begin{table}[H]
  \centering
    \begin{adjustbox}{max width=\textwidth}
  \begin{tabular}{rp{14cm}}
    \toprule
    \textbf{ID} & \uselabel{use:UseCaseSignUpUser}\\
    \hline
    \textbf{Name} & Sing up of Users.\\
    \hline
    \textbf{Actor} & User.\\
    \hline
    \textbf{Entry conditions} & The user must have the application installed and be in it.\\
    \hline
    \textbf{Events flow} &
    \begin{tabular}{@{}p{14cm}@{}}
        \begin{enumerate}[leftmargin=*]
            \item The user have to click on the button of sing up of the application (figure \ref{fig:LogInScreenApp}).
            \item Fill in the necessary information requested in the form that will appear.
            \item After the confirmation the system will save the data and the user will be registered.
        \end{enumerate} 
    \end{tabular} \\ 
    \hline
    \textbf{Exit conditions} & The user will be registered and able to use \companyName{} services.\\
   
    \hline
    \textbf{Exceptions} & 
     \begin{tabular}{@{}p{14cm}@{}}
        \begin{enumerate}[leftmargin=*]
            \item The username already exists in the system.
            \item The email already exists in the system.
            \item The user has not filled in one of the necessary information fields or a field is not filled in correctly.
        \end{enumerate} 
    \end{tabular} \\
    
    \bottomrule
  \end{tabular}
  \end{adjustbox}
  \caption{Sing up of a User use case}
  \label{table:UseCaseSingUp}
\end{table}

\begin{table}[H]
  \centering
    \begin{adjustbox}{max width=\textwidth}
  \begin{tabular}{rp{14cm}}
    \toprule
    \textbf{ID} & \uselabel{use:UseCaseLogInUser}\\
    \hline
    \textbf{Name} & Log in of Users.\\
    \hline
    \textbf{Actor} & User.\\
    \hline
    \textbf{Entry conditions} & 
    \begin{tabular}{@{}p{14cm}@{}}
         \begin{enumerate}
             \item The user have to be registered on \companyName{}.
             \item The user needs to have the application installed.
         \end{enumerate}  \\
    \end{tabular}\\
    \hline
    \textbf{Events flow} &
    \begin{tabular}{@{}p{14cm}@{}}
        \begin{enumerate}[leftmargin=*]
            \item Press the log in button in the application (figure \ref{fig:LogInScreenApp}).
            \item Complete the username and password sections of the log in window.
            \item After clicking on the log in button, if it is correct the username and password the user will access be redirected to the main window (figure \ref{fig:PrincipalScreenApp}).
        \end{enumerate} 
    \end{tabular} \\ 
    \hline
    \textbf{Exit conditions} & The user will access his account.\\
   
    \hline
    \textbf{Exceptions} & Username and password do not match or do not exist.\\
    
    \bottomrule
  \end{tabular}
  \end{adjustbox}
  \caption{Log in of a User use case}
  \label{table:UseCaseLogInUser}
\end{table}

\begin{table}[H]
  \centering
    \begin{adjustbox}{max width=\textwidth}
  \begin{tabular}{rp{14cm}}
    \toprule
    \textbf{ID} & \uselabel{use:UseCaseIndividualSearch}\\
    \hline
    \textbf{Name} & Search of an individual.\\
    \hline
    \textbf{Actor} & Client.\\
    \hline
    \textbf{Entry conditions} & The client have to be registered on \companyName{} and log in on the web application.\\
    \hline
    \textbf{Events flow} &
    \begin{tabular}{@{}p{14cm}@{}}
        \begin{enumerate}[leftmargin=*]
            \item Click on the individual data search button.
            (Figure \ref{fig:PrincipalPageWebApp})
            \item Enter the Codice Fiscale in the search area that appears on the new page (in the area where it is requested). (Figure \ref{fig:SearchIndividualWebApp})
            \item The system will show the data of the user if the client have the necessary permissions, another search can be performed also.
           (Figures \ref{fig:ResultIndividualSearch1}, \ref{fig:ResultIndividualSearch2}, \ref{fig:ResultIndividualSearch3})
        \end{enumerate} 
    \end{tabular} \\ 
    \hline
    \textbf{Exit conditions} & The client will be able to see the information of the requested user.\\
   
    \hline
    \textbf{Exceptions} & 
    \begin{tabular}{@{}p{14cm}@{}}
         \begin{enumerate} [leftmargin=*]
             \item The Codice Fiscale does not exist.
             \item The Codice Fiscale is misspelled.
             \item The client does not have the permissions to view the user's data.
         \end{enumerate}  \\
    \end{tabular}\\
    
    \bottomrule
  \end{tabular}
  \end{adjustbox}
  \caption{Case of use of individual data search }
  \label{table:UseCaseIndividualSearch}
\end{table}

\begin{table}[H]
  \centering
    \begin{adjustbox}{max width=\textwidth}
  \begin{tabular}{rp{14cm}}
    \toprule
    \textbf{ID} & \uselabel{use:UseCaseGroupSearch}\\
    \hline
    \textbf{Name} & Querying group data.\\
    \hline
    \textbf{Actor} & Client.\\
    \hline
    \textbf{Entry conditions} & The client have to be registered on \companyName{} and log in on the web application.\\
    \hline
    \textbf{Events flow} &
    \begin{tabular}{@{}p{14cm}@{}}
        \begin{enumerate}[leftmargin=*]
            \item Click on the group data search button of the web application (figure \ref{fig:PrincipalPageWebApp}).
            \item Client will be redirected to a search page, figure \ref{fig:GroupSearch}, where a query can be formulated.
            \item After pressing the search button the system will show the information of the users (anonymously) who meet the criteria given (figures \ref{fig:ResultGroupSearch}, \ref{fig:ResultGroupSearch2} and \ref{fig:ResultGroupSearch3}).
        \end{enumerate} 
    \end{tabular} \\ 
    \hline
    \textbf{Exit conditions} & The client will be able to see the data of the group that fulfills the given requirements.\\
   
    \hline
    \textbf{Exceptions} & 
    \begin{tabular}{@{}p{14cm}@{}}
         \begin{enumerate} [leftmargin=*]
             \item Insert a search criteria that the set of users that satisfy them is less than 1000.
             \item Any of the search criteria given is misspelled or does not exist.
         \end{enumerate}  \\
    \end{tabular}\\
    
    \bottomrule
  \end{tabular}
  \end{adjustbox}
  \caption{Case of use of group data search }
  \label{table:CaseUseGroupSearch}
\end{table}

The following state charts are mean to clarify the operation of the system.

\begin{figure}[H]
  \centering
   \includegraphics[width=0.65\textwidth]{Images/StatechartBusquedaGrupal}
  \caption{Statechart group search}
  \label{fig:StatechartGroupSearch}
\end{figure}

In figure \ref{fig:StatechartGroupSearch} it can be saw the state chart diagram of the group search which corresponds to use case \useref{table:CaseUseGroupSearch} and figure \ref{fig:GroupSearch}.

\begin{figure}[H]
  \centering
   \includegraphics[width=0.65\textwidth]{Images/StatechartIndividualSearch}
  \caption{Statechart individual search}
  \label{fig:StatechartIndividualSearch}
\end{figure}

In figure \ref{fig:StatechartIndividualSearch} it can be saw the state chart diagram of the group search which is assigned to the use case \useref{table:UseCaseIndividualSearch} and figures \ref{fig:SearchIndividualWebApp}, \ref{fig:ResultIndividualSearch1}, \ref{fig:ResultIndividualSearch2} and \ref{fig:ResultIndividualSearch3}.

The following sequence diagram attempts to show the main workings of the application for better understanding.

\begin{figure}[H]
  \centering
   \includegraphics[width=0.95\textwidth]{Images/TrackSequenceDiagram}
  \caption{Track sequence diagram}
  \label{fig:TrackSequenceDiagram}
\end{figure}

Figure \ref{fig:TrackSequenceDiagram} shows the sequence diagram of the tracking carried out by the application each time interval and as in the case of parameters below what is necessary, an alarm is sent to the user and to the emergency services.

The following class diagram summarizes how the system is organized and what main structures it contains.

\begin{figure}[H]
  \centering
   \includegraphics[width=0.95\textwidth]{Images/ClassDiagram}
  \caption{Class diagram of the system}
  \label{fig:ClassDiagram}
\end{figure}

Figure \ref{fig:ClassDiagram} shows the main structures of the system as they are related to each other and how the main functions (Search and Track) are related to the system to be performed.

\subsection{Functional Requirements}
The functional requirements are divided in three categories. The requirements of the user application, table \ref{table:user-reqs}; the requirements related with clients, table \ref{table:client-dashboard-reqs} and \ref{table:client-api-reqs}, and finally the AutomatedSOS requirements, table \ref{table:automatedsos-reqs}. 

\begin{table}[H]
  \centering
    \begin{adjustbox}{max width=\textwidth}
  \begin{tabular}{ccp{14cm}}
    \toprule
    \textbf{ID} & \textbf{Goal} & \textbf{Description} \\
    \midrule
    
    \reqlabel{req:welcome} &
    \goalref{gl:registration} & \begin{tabular}{@{}p{14cm}@{}}
    When an user opens the application and no login had been performed, the system shall show the welcome page (figure \ref{fig:LogInScreenApp}).
    \end{tabular} \\
    \hline
    
    \reqlabel{req:login} &
    \goalref{gl:registration} & \begin{tabular}{@{}p{14cm}@{}}
    When the welcome page is shown, the system shall show two buttons (figure \ref{fig:LogInScreenApp}). When clicked, one of them shall redirect to the login page and the other to the registration page. 
    \end{tabular} \\
    \hline
    
    \reqlabel{req:terms-conditions} &
    \goalref{gl:registration} & \begin{tabular}{@{}p{14cm}@{}}
    When the registration page is completed, the system shall show the terms and conditions page and only users that accept the terms and conditions will successfully registered.
    \end{tabular} \\
    \hline
    
    \reqlabel{req:recollection-data-permissions} &
    \goalref{gl:how-recollect} & \begin{tabular}{@{}p{14cm}@{}}
    When the user logs in for the first time in the application, the application shall check what sensors are available an issue an Android Permission Request for each of them.
    \end{tabular} \\
    \hline
    
    \reqlabel{req:recollection-data-permissions-no} &
    \goalref{gl:how-recollect} & \begin{tabular}{@{}p{14cm}@{}}
    If the user declines an Android Permission Request, the application shall issue again an Android Permission Request for the same sensor.
    \end{tabular} \\
    \hline
    
    \reqlabel{req:recollection-data-functioning} &
    \goalref{gl:how-recollect} \goalref{gl:intervals} & \begin{tabular}{@{}p{14cm}@{}}
    The system shall poll the available sensors in the background at fixed time intervals and store the measures in the server. 
    \end{tabular} \\
    \hline
    
    \reqlabel{req:recollection-data-intervals} &
    \goalref{gl:intervals} & \begin{tabular}{@{}p{14cm}@{}}
    The fixed intervals at which each sensor shall be polled are stated in \ref{table:parameters-intervals}.
    \end{tabular} \\
    \hline
    
    \reqlabel{req:recollection-data-maual-functioning} &
    \goalref{gl:how-recollect} \goalref{gl:intervals} & \begin{tabular}{@{}p{14cm}@{}}
    The system shall prompt the user to introduce the manual parameters at fixed time intervals and store the measures in the server. 
    \end{tabular} \\
    \hline
    
    \reqlabel{req:recollection-manual-data-intervals} &
    \goalref{gl:intervals} & \begin{tabular}{@{}p{14cm}@{}}
    The fixed intervals at which the manual parameters shall be asked to the user are stated in \ref{table:parameters-intervals}.
    \end{tabular} \\
    \hline
    
    \reqlabel{req:allow-individual-tracking} &
    \goalref{gl:individuals} & \begin{tabular}{@{}p{14cm}@{}}
    When a client has sent a request for access, the system shall display a notification in the user's device showing the name of the client which requires the permission and a button to accept.
    \end{tabular} \\
    
    \bottomrule
  \end{tabular}
  \end{adjustbox}
  \caption{Functional requirements of user application}
  \label{table:user-reqs}
\end{table}

\begin{table}[H]
  \centering
    \begin{adjustbox}{max width=\textwidth}
  \begin{tabular}{ccp{14cm}}
    \toprule
    \textbf{ID} & \textbf{Goal} & \textbf{Description} \\
    \midrule
    
    \reqlabel{req:query-format} &
    \goalref{gl:query} & \begin{tabular}{@{}p{14cm}@{}}
    A query consists of a set of parameters with associated logical constraints. The result of the query must comply all the logical constraint in the query.
    \end{tabular} \\
    \hline
    
    \reqlabel{req:logical-constraints-format} &
    \goalref{gl:query} & \begin{tabular}{@{}p{14cm}@{}}
    The numerical parameters' logical constraints can be equal (=), greater (>), greater or equal (=>), smaller (<) and smaller or equal (=<). The \paramref{pm:location} parameter do not follow this requirement.
    \end{tabular} \\
    \hline
    
    \reqlabel{req:location-query} &
    \goalref{gl:query} & \begin{tabular}{@{}p{14cm}@{}}
    The \paramref{pm:location} parameter's logical constraint is expressed as a set of points in which the searched values are geographically inside.
    \end{tabular} \\
    \hline
    
    \reqlabel{req:easy-query} &
    \goalref{gl:query} & \begin{tabular}{@{}p{14cm}@{}}
    The system should provide specific inputs adapted to the type of data to introduce the logical constraints of the query. Table \ref{table:parameters-inputs} states the parameters and its inputs.
    \end{tabular} \\
    \hline
    
    \reqlabel{req:client-query} &
    \goalref{gl:query} \goalref{gl:privacy}& \begin{tabular}{@{}p{14cm}@{}}
    When the client introduces a query from the dashboard page (figure \ref{fig:GroupSearch}) and the number of entries that fullfil the query are equal or more than 1000, the system shall show the data in page (Figures \ref{fig:ResultGroupSearch}, \ref{fig:ResultGroupSearch2} and \ref{fig:ResultGroupSearch3}).
    \end{tabular} \\
    \hline
    
    \reqlabel{req:client-query-few} &
    \goalref{gl:query} \goalref{gl:privacy}& \begin{tabular}{@{}p{14cm}@{}}
    When the client introduces a query from the dashboard page (figure  \ref{fig:GroupSearch}) and the number of entries that fullfil the query are less than 1000, the system shall warn the client about the impossibility to show the results in page.
    \end{tabular} \\
    \hline
    
    \reqlabel{req:query-live} &
    \goalref{gl:real-time} & \begin{tabular}{@{}p{14cm}@{}}
    When the system is showing a data batch in the dashboard that fullfils a query (figures individual search: \ref{fig:ResultIndividualSearch1}, \ref{fig:ResultIndividualSearch2} and \ref{fig:ResultIndividualSearch3}; figures group search: \ref{fig:ResultGroupSearch}, \ref{fig:ResultGroupSearch2} and \ref{fig:ResultGroupSearch3}) and new data that also fullfils the query arrives, the system shall update the view of the data without intervention of the client.
    \end{tabular} \\
    \hline
    
    \reqlabel{req:client-query-one} &
    \goalref{gl:individuals} & \begin{tabular}{@{}p{14cm}@{}}
    When the client introduces a codice fiscale from the dashboard page (figure \ref{fig:SearchIndividualWebApp}), the user exists and the client have already obtained the permission of the user, the system shall return the data associated to the individual.
    \end{tabular} \\
    \hline
    
    \reqlabel{req:client-codice-permission} &
    \goalref{gl:individuals} & \begin{tabular}{@{}p{14cm}@{}}
    When the client introduces a codice fiscale from the dashboard page (figure \ref{fig:SearchIndividualWebApp}), the user exists and the client do not have the permission of the user, the system shall prompt the client to ask permission to the user.
    \end{tabular} \\
    \hline
    
    \reqlabel{req:client-codice-noexist} &
    \goalref{gl:individuals} & \begin{tabular}{@{}p{14cm}@{}}
    When the client introduces a codice fiscale from the dashboard page (figure \ref{fig:SearchIndividualWebApp}), and the user do not exists, the system shall prompt the client to ask permission to the user.
    \end{tabular} \\
    \hline
    
    \reqlabel{req:client-ask-permissions} &
    \goalref{gl:individuals} & \begin{tabular}{@{}p{14cm}@{}}
    When the client is requesting permission to a concrete user in page and clicks on \textit{Yes}, the system shall emit a to the appropriate user application requesting their permission.
    \end{tabular} \\
    \hline
    
    \reqlabel{req:client-gets-permission} &
    \goalref{gl:individuals} & \begin{tabular}{@{}p{14cm}@{}}
    When a user approves the request of access made by a client, the system shall store that permission.
    \end{tabular} \\
    \hline
    
    \reqlabel{req:client-list-permissions} &
    \goalref{gl:individuals} & \begin{tabular}{@{}p{14cm}@{}}
    The system shall show the client a list of all users that had give their permission of access in page in descending alphabetical order.
    \end{tabular} \\
    
    \bottomrule
  \end{tabular}
  \end{adjustbox}
  \caption{Functional requirements of the client dashboard}
  \label{table:client-dashboard-reqs}
\end{table}

In table \ref{table:client-api-reqs} the phrase \textit{When a message with a correct format reach} is used often. The correct format reefers to the one stated in section \ref{sec:interfaces} for each corresponding SI interface.

\begin{table}[H]
  \centering
    \begin{adjustbox}{max width=\textwidth}
  \begin{tabular}{ccp{14cm}}
    \toprule
    \textbf{ID} & \textbf{Goal} & \textbf{Description} \\
    \midrule
    \reqlabel{req:api-login} &
    \goalref{gl:registration} & \begin{tabular}{@{}p{14cm}@{}}
    When a message with a correct format reach the interface \siref{si:api-login} with an existing pair of username and password, the system shall replay with a token that will identify the client in the next api calls. The token have a validity of 3 days.
    \end{tabular} \\
    \hline
    
    \reqlabel{req:api-query} &
    \goalref{gl:query}, \goalref{gl:privacy} & \begin{tabular}{@{}p{14cm}@{}}
    When a message with a correct format reach the interface \siref{si:api-query} with a well form query and the result of the query have 1000 entries or more, the system shall replay with a data batch that complies the logical constraints expressed in the query.
    \end{tabular} \\
    \hline
    
    \reqlabel{req:api-query-few} &
    \goalref{gl:query}, \goalref{gl:privacy} & \begin{tabular}{@{}p{14cm}@{}}
    When a message with a correct format reach the interface \siref{si:api-query} with a well form query and the result of the query have less than 1000 entries, the system shall replay with a 403 error.
    \end{tabular} \\
    \hline
    
    \reqlabel{req:api-search} &
    \goalref{gl:individuals} & \begin{tabular}{@{}p{14cm}@{}}
    When a message with a correct format reach the interface \siref{si:api-search} with a valid codice fiscale, the user exists and the client have already obtained the permission of the user, the system shall return the data associated to the individual.
    \end{tabular} \\
    
    \bottomrule
  \end{tabular}
  \end{adjustbox}
  \caption{Functional requirements of the client API}
  \label{table:client-api-reqs}
\end{table}

\begin{table}[H]
  \centering
    \begin{adjustbox}{max width=\textwidth}
  \begin{tabular}{ccp{14cm}}
    \toprule
    \textbf{ID} & \textbf{Goal} & \textbf{Description} \\
    \midrule
    \reqlabel{req:automated-sos-client} &
    \goalref{gl:emergency} & \begin{tabular}{@{}p{14cm}@{}}
    When a parameter sent by an user's application arrives at the server and is below a defined threshold and the user is sign up in AutomatedSOS, the system shall rise an alarm to the Emergency System using interface \siref{si:emergency-alarm} within 5 seconds.
    \end{tabular} \\
    
    \bottomrule
  \end{tabular}
  \end{adjustbox}
  \caption{Functional requirements of the AutomatedSOS service}
  \label{table:automatedsos-reqs}
\end{table}


\subsection{Performance Requirements}

\begin{table}[H]
  \centering
    \begin{adjustbox}{max width=\textwidth}
  \begin{tabular}{cp{14cm}}
    \toprule
    \textbf{ID} & \textbf{Description} \\
    \midrule
    \perflabel{perf:req-per-second} & \begin{tabular}{@{}p{14cm}@{}}
    The system shall be able to process at least 500 request per second.
    \end{tabular} \\
    \hline
    \perflabel{perf:response-time} & \begin{tabular}{@{}p{14cm}@{}}
    The system shall be able to respond to a query made by a client within 5 seconds in an optimum internet connection between client and system scenario.
    \end{tabular} \\
    \hline
    \perflabel{perf:storage-capacity} & \begin{tabular}{@{}p{14cm}@{}}
    The system shall be able to store all the data gathered from the users at least during 3 months. 
    \end{tabular} \\
    \hline
    \perflabel{perf:number-of-users} & \begin{tabular}{@{}p{14cm}@{}}
    The system shall be able to handle at least 10000 active users.
    \end{tabular} \\
    \hline
    \perflabel{perf:number-of-clients} & \begin{tabular}{@{}p{14cm}@{}}
    The system shall be able to handle at least 100 active clients.
    \end{tabular} \\
    
    \bottomrule
  \end{tabular}
  \end{adjustbox}
  \caption{Performance requirement}
  \label{table:performance-reqs}
\end{table}


\subsection{Design Constraints}

\subsubsection{Standards compliance}
Since \companyName{} system do not have to handle any critical situation nor have connection with the real world, the applicability of standards is limited. 

Although law is not a standard strictly speaking, since the impact of the rules about privacy and usage of data is noticeable in \companyName{} product, a review of the applicable legislation must be made.

Since May 2018, the EU General Data Protection Regulation replaces the previous legislation, enforcing a homogeneous set of rules in the whole Europe Union. The following excerpt is obtained from \cite{GPDR}:

\textit{The request for consent must be given in an intelligible and easily accessible form, with the purpose for data processing attached to that consent. Consent must be clear and distinguishable from other matters and provided in an intelligible and easily accessible form, using clear and plain language. It must be as easy to withdraw consent as it is to give it.}

Therefore, the registration process must be clear, stating in a natural language the terms of usage. Users needs to accept one per one the collection of each peace of information that \companyName{} is collecting. Users also can exercise their right to withdraw their consent. A section in the users' application offering to withdraw their permission must be present. Please, notice the Privacy section in figure \ref{fig:PrincipalScreenAppWithSlide}.

\subsubsection{Hardware limitations}

There are no hardware limitations, it is only necessary for the user to present a mobile phone with the possibility of installing mobile applications (a smartphone) and if a health check is required, the user must have a Smartwatch.

In addition to what has been said for the mobile phone, it is also necessary that the user has:
 \begin{itemize}
        \item Stable Internet connection (2G, 3G, 4G).
        \item Bluetooth for connection to the Smartwatch if required.
        \item GPS for user location.
 \end{itemize}
 
The Smartwatch the only requirement is to be able to measure the heart rate as well as the different parameters to perform health monitoring.

\subsection{Software System Attributes}

\subsubsection{Reliability}
The system shall preserve the integrity of the saved data no matter what circumstances may happen.
\subsubsection{Availability}
The system shall be available 99.99\% of the time. Programmed maintenance downtime communicated to \companyName{} on behalf of JJ Software at least one month ahead will not account in the availability time.

In the event that an external factor causes \companyName{} services to fall, planned emergency measures will be taken to ensure that the system continues to operate at all times.

\subsubsection{Security}
For security, the HTTPS protocol will be used, which applies SSL/TLS cryptography \cite{HTTPS} that will allow the passage between the server and the different services in a secure way, not allowing attacks or loss of information from \companyName{} clients and users.

\subsubsection{Maintainability}
The operation and maintenance of the \companyName{} system will be realised by JJ Software under the terms agreed on the Maintenance and Operation contract.

The routine tasks as minor fixes will be performed with no charge to \companyName{}. Minor fixes includes light tasks with a budget time of two hours of work monthly.

One important aspect to highlight is the Terms of Conditions that will be presented to users and clients. The Terms of Conditions shall be easy to change and update in order to adapt the \companyName{} system to legislation changes.

\subsubsection{Portability}
The portability degree depends on each system's component. The mobile application shall be portable between Android platforms that fullfills the requirements stated in table \ref{table:mobile-constraints}. The server component shall be portable between Linux platforms that fullfills the requirements stated in table \ref{table:server-constraints}. The online dashboard  shall work in any browser that fullfills the rules stated in table \ref{table:dashboard-constraints}
