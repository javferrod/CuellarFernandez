
\subsection{Purpose}
The purpose of this document is to give a detailed specification for the \companyName{} software product. Along the following pages the goals, requirements, constraints and interfaces of the project will be explained. The intention of this document is not only to be proposed to the customer but to be used as the ground for the development of the product.

This document could be used as a contractual basis for the realisation of the project. 

\companyName{} enhances the flowing of data from the users to the clients, enabling the companies to make right choices about their users thanks to the analysis of the data. These data may encompass location, heart rate, age and so. The data needs to be recollected in a time basis, allowing the clients to analyse their evolution trough the time.

\subsection{Scope}

The recollection of the data should be carried out in an automated fashion, using the sensors available in mobile devices such as mobiles (smartphones) and smart-watches. 

Such automation should not undermine the privacy rights of the users. The system should provide mechanism for the users to grant or deny their approval for the recollection and treatment of their data.

The main goal of the system is to provide tools for the analysis of the recollected data. Therefore, the system should provide a dashboard and an API to the clients, allowing them to navigate and query the available data.

%The recollection of the data will be carried out by the mobile devices of the users, upon installation of the application of \companyName{}. Before the recollection of data takes place, the users have to register to the system and give their approval.

%To enable the analysis of the data to the clients, \companyName{} provides centralised tools by means of an online dashboard and an API endpoint. The latter is intended to facilitate the integration of \companyName{} platform with client's ones.

%Privacy issues that may arise. The recollection of private data is subject to a strong regulation in the European Union and therefore, mechanism to enforce a fair usage of the platform may be in place.

%As a general rule, the querying of information should never return data that enable the differentiation of individuals. Returning a few records for a query may expose the individuals behind that data, and therefore is not allowed.

%The only option for a client to access to individual data is to request permission to the specific user. This will be done by mean of the system. The client should provide the unique identifier of the user, for example the codice fiscale. 

The goals of the system are summarised in table \ref{table:goals}.

\begin{table}[H]
  \centering
    \begin{adjustbox}{max width=\textwidth}
  \begin{tabular}{cp{14cm}}
    \toprule
    \textbf{ID} & \textbf{Goal} \\
    \midrule
    
    \goallabel{gl:registration} & \begin{tabular}{@{}p{14cm}@{}}
    The system should provide accounting and authorisation for users and clients.
    \end{tabular} \\
    \hline
    \goallabel{gl:recollect} & \begin{tabular}{@{}p{14cm}@{}}
    The system should store the recollected data.
    \end{tabular} \\
    \hline
    
    \goallabel{gl:how-recollect} & \begin{tabular}{@{}p{14cm}@{}}
    The system should recollect the data using the sensors available in the users' devices, asking the user directly the information when no sensor is available for recollecting the information (for example, weight). 
    \end{tabular} \\
    \hline
    
    \goallabel{gl:intervals} & \begin{tabular}{@{}p{14cm}@{}}
    The system should recollect the data from the users at time intervals.
    \end{tabular} \\
    \hline
    
    \goallabel{gl:time} & \begin{tabular}{@{}p{14cm}@{}}
    The system should store and display the data in a time series format, allowing the client to consult the changes in the parameters along the time.
    \end{tabular} \\
    \hline
    
    \goallabel{gl:query} & \begin{tabular}{@{}p{14cm}@{}}
    The system should allow the clients to easily query the already recollected data of the users. 
    \end{tabular} \\
    \hline
    
    \goallabel{gl:individuals} & \begin{tabular}{@{}p{14cm}@{}}
    The system should allow the clients to query the data of an specific user.
    \end{tabular} \\
    \hline
    
    \goallabel{gl:real-time} & \begin{tabular}{@{}p{14cm}@{}}
    The system should allow the clients to subscribe to a query, providing new data as arrives.
    \end{tabular} \\
    \hline
    
    \goallabel{gl:privacy} & \begin{tabular}{@{}p{14cm}@{}}
    The system should protect the privacy of the users. A data batch displayed to a client should not enable the differentiation between individuals.
        \end{tabular} \\
    \hline
    \goallabel{gl:emergency} & \begin{tabular}{@{}p{14cm}@{}}
    The system should allow users to monitor some of their parameters, alerting the emergency system when any of these parameter gets out of a threshold.
        \end{tabular} \\
    \bottomrule
  \end{tabular}
  \end{adjustbox}
  \caption{Goals}
  \label{table:goals}
\end{table}

\subsection{Definitions, acronyms and abbreviations}
\subsubsection{Definitions}\label{sec:definitions}
\begin{itemize}
    \item \textbf{Administrator:} Worker of \companyName{} with access to the entire platform without restrictions.
     
    \item \textbf{Alarm:} A warning given to the user and emergency services when health parameters are lower than necessary.
  
    \item \textbf{Client:} Individual or company that pays \companyName{} to get access to the data of the users of \companyName{}.
    
    \item \textbf{Data batch:} Collection of data from different users that comply with a query based on logical constraints made by a client.
    
    \item \textbf{Emergency system:} External system that performs the emergency warning functions for the AutomatedSOS service.
    
    \item \textbf{JJ Software:} Software company responsable of the implementation, operation and maintenance of \companyName{} software.
    
    \item \textbf{Logical constraint:} Boolean sentence made upon a Parameter, the queries are formulated as a set of constraints. The data batch ensuing of a query complies with all the logical constraints of the query.
    
    \item \textbf{Measure:} Data obtained by the users' devices and communicated to the server. It is classified in parameters.
    
    \item \textbf{Optimum internet connection:} Connection with at least 10 Mb/s and a latency below 20 ms.
    
    \item \textbf{Parameter:} Type of data in which the recollected data is organised. These parameters may include blood pressure, location.
    
    \item \textbf{Payment system:} External system that allows us to make the different payments to clients.
    
    \item \textbf{Plan ID:} Unique identifier of a Plan. A Plan is  an object of Stripe which represents a monthly cost associated to the services offered by \companyName{}.
    
    \item \textbf{Subscription ID:} Unique identifier of a Subscription. A Subscription is an object of Stripe representing the association of a Client with a Plan. When a subscription is created, the client is charged the amount of the Plan in a monthly basis.
    
    \item \textbf{Android Permission Request:} Message presented to an Android user requesting allowance of the user to collect data from some sensor, as the location.
    
    \item \textbf{Track:} This is the tracking of data, people or any type of item that can be traced and tracked.
    
    \item \textbf{Token:} Unique identifier which replace username and password, have a limited duration in time. 
    
    \item \textbf{User:} Individual that installs the \companyName{} application and give \companyName{} permission to collect and sell their personal data.
    
\end{itemize}
\subsubsection{Acronyms}
\begin{itemize}
    \item \textbf{API}: Application Program Interface.
    \item \textbf{GPS}: Global Positioning System.
    \item \textbf{HTTPS}: Hyper Text Transfer Protocol Secure.
    \item \textbf{SSL}: Secure Sockets Layer.
    \item \textbf{TLS}: Transport Layer Security.
\end{itemize}
\subsubsection{Abbreviations}
\begin{itemize}
    \item \textbf{403 error}: Forbidden HTTP error code.
    \item \textbf{bpm}: Beats per minute.
    \item \textbf{CVC}:Card Verification Code
    \item \textbf{M/F}: Male/Female.
\end{itemize}

\subsection{Revision history}

\begin{itemize}
    \item \textbf{V 1.0:} First version of the document directed at consumers and developers.
\end{itemize}



\subsection{Reference Documents}

The various pages and documents referred to in this document can be found in page \pageref{sect:references}.

During the reading of the document you will find notation to the different references that have been used.

\subsection{Document Structure}

The document has been structured in different sections and subsections as can be seen on page \pageref{sect:contents}.

During this section we will try to explain in more detail what is dealt with in each section or subsection.

The document is divided into 5 large sections which are divided into multiple subsections, some subsections are further divided into subsections.

\begin{itemize}
    \item The first section, is based on an introduction in which talk about the idea that is followed to carry out this document and the project, the objectives that are wanted to reach and that must fulfill the final project as well as different definitions, acronyms or abbreviations that are going to be found throughout the project so that no doubt is generated in the reader.
    
    This section begins on page \pageref{sect:introduction}.
    
    \item The second section make a more in-depth perspective of the product by treating in a very exact way which users the application is going to be aimed at, dividing them into different groups to name and observe in a more detailed way their different characteristics, we observe the different functions that the project is going to present and how these are fulfilling the different objectives, also observe in detail what external systems are going to be used and what external problems are found to fulfill these objectives.
    
    This section begins on page \pageref{sect:overview}.

    \item The third section gets more into how the application is inside, leaving behind the aerial view of the previous sections.
    In this section we talk about how the application will look using different designs to see the final idea of both the mobile application and the web application as well as the smartwatch. It also deals with the hardware and software interfaces as well as the different requirements expected from the project at a functional and performance level.
    It presents the reader with different use cases and diagrams so that he or she has no doubt when using the different services, it deals with the different limitations observed when trying to adapt the applications and services to the different devices and finally it is observed how the system is going to be maintained at the level of security, maintenance, portability and different factors.
    
    This section begins on page \pageref{sect:requirements}.
    
    \item The fourth section contains the alloy model of the services which will be used in their future development.
    
    This section begins on page \pageref{sect:alloy}.
    
    \item The last section contains the times used for the development of this document.
    
    This section begins on page \pageref{sect:effort}.
    
\end{itemize}    
    
    At the end of the document the references used are included as well as two appendices on the parameters and intervals used by \companyName{}.
    


