\subsection{API}
The framework used is Koa.js. Koa.js is based on Express, which is well established as framework for JavaScript based APIs. The advantages of the former over the latter is the minimalist approach and the support to ES2017 functionality.

The database chosen is Timescale, a relative new database which reconciles SQL with timeseries data. Since its newness, there are not specific libraries to deal with the connection between database and backend. To tackle this issue, knex.js is used. 

Knex.js provides a thin client with plays well with Postgres, in which timescale is based. Although it do not have fancy features, Knex.js provides a fine-grained control over the SQL statements executed and, in the worst case, a fallback to raw SQL\footnote{Which is used to build the hypertables of Timescale}.

\subsubsection{Build and runtime}
Javascript is well known for causing numerous headaches in terms of creating the necessary environment for its execution. Babel is used to transpile the code into something that NodeJS can understand and yarn/npm is the option chosen to manage the dependencies. 

The runtime is handled by Docker and docker-compose. The former takes care of the virtualisation of the containers whilst the latest carry on the orchestration between them. 

There are two containers, one for the database and another for NodeJS. The database container can be obtained automatically by docker\footnote{The image is timescale/timescaledb, can be found in dockerhub} but the NodeJS one needs to be build locally, since the API code is packed inside.

\subsection{Frontend}
React is used to bring the user interface to life. Today, considered one of the kings as far as frontend is concerned. Redux is employed to handle the state of the application.

There are two main advantage in the approach suggested by Redux: centralisation and immutability. The former helps in controlling the state that would otherwise be scattered over various components. The latter plays well with React, that needs immutability to compute the differences and render the components correspondingly.

\subsubsection{Build and runtime}
As in the API, Babel is in charge of the transpilation. The code generated by Babel is supported by the versions of the browser indicated in the RASD. 

The building process is handled by parcel, which provides HRM\textit{Hot Module Replacement}, multicore and cached compilation.

Worth mentioning that in a production application the code will be minified in a single file. This file will be included in a Docker container in front of the API one, serving the frontend. NginX may be a good option for this task, as excels serving static files.
\subsubsection{}

\subsection{Mobile Application}
For the operation of the application applies Java and XML for Android, thanks to the union of these two languages can get an excellent job thanks to the great potential they present.

